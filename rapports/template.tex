\documentclass[12pt]{article}
\usepackage[french]{babel}  % Pour le français
\usepackage[utf8]{inputenc} % Pour taper les caractères accentués
\usepackage{amsmath}  % Ces trois paquets donnent accès à 
\usepackage{amsfonts} % des symboles et formulations
\usepackage{amstext}  % mathématiques
\usepackage{hyperref} % Permet de faire automatiquement des liens dans les
                      % documents
\usepackage{graphicx} % Permet d'insérer des images
\usepackage[utf8]{inputenc}
\usepackage{geometry}
\usepackage{graphics}
\usepackage[T1]{fontenc}
\usepackage[frenchb]{babel}
\usepackage{amsmath}
\usepackage{amssymb}
\usepackage{mathrsfs}
\usepackage{graphicx}
\usepackage{caption}
\usepackage{float}
\usepackage{amsfonts}
\usepackage{enumitem}
\usepackage{pifont}
\usepackage{array}
%\usepackage{mathbb}
\addto\captionsfrench{\def\tablename{TABLEAU}}
\geometry{hmargin=2cm,vmargin=2cm}
\DeclareGraphicsExtensions{.png} % Mettre ici la liste des extensions des
                                 % fichiers images

% On peut choisir la police en utilisant un paquet 
%\usepackage{newcent}
\usepackage{lmodern}
%\usepackage{cmbright} % Computer Modern Bright

% Une des nombreuses manière de modifier les marges par défaut
\usepackage{geometry}
\geometry{vmargin=2cm,hmargin=2.5cm,nohead}

% On peut redéfinir certaines longueurs, par exemple l'espacement entre les
% paragraphes:
\setlength{\parskip}{0.25cm}

% Quelques définitions 
\def \rr {{\mathbb R}} % L'ensemble R
\def \cc {{\mathbb C}} % L'ensemble C
\def \nn {{\mathbb N}} % L'ensemble N
\def \zz {{\mathbb Z}} % L'ensemble Z

% Les informations de la page de titre (page de titre séparée pour un 'report').
\title{Mon premier rapport en \LaTeX}
\author{
  Moi Même\thanks{\href{mailto:yves.coudiere@u-bordeaux.fr}{\tt yves.coudiere@u-bordeaux.fr}}
  \and
  Qq'un d'autre\thanks{Bla bla}
}
% \thanks: permet de mettre une note de bas de page pour l'auteur
% \href: insère un lien, ici vers l'application 'mailto'
% \tt: police monospace
\date{\today} % \today pour la date courante 

\begin{document}

\begin{titlepage}
\begin{center}

%\includegraphics[width=8cm]{Logo_EIRB.PNG}

\vspace{3cm}

\huge
\textbf{Travaux Pratiques} \\
\vspace{0.7cm}
\Large
Compte Rendu Hebdomadaire TP2\\
\ Début travail en groupe \\
\large
\vspace{0.7cm}
M1 - EDPMA\\
Groupe AlMati\\
\vspace{0.7cm}
Alice \bsc{Castagnet} \\ Marine \bsc{Redondo} \\ Tiffanie \bsc{Carlier} \\
\vspace{0.7cm}
Encadrant\\
\bsc{M. Leguèbe}\\
\vspace{2cm}
\Large
\textbf{2 octobre 2018}
\end{center}
\end{titlepage}

\normalsize

\newpage


\section{Première partie : recherche de ressources, documentation}

1. Une équation de transport est une équation de la forme :

\begin{eqnarray}
\partial_tu(t,x)+c(t,x,u).\nabla_xu(t,x)=0
\end{eqnarray}

Elle décrit le comportement d'une quantité $u$ dépendant du temps $t$ et d'une autre variable $x$ appartenant à un domaine $\Omega$ qui représente généralement un ouvert régulier de $\mathbb{R}^n$. La variable $x$ peut définir une position, une vitesse, un couple position/vitesse ... Les équations de transport font apparaître une notion de propagation à vitesse finie et sont donc qualifiées d'équations d'hyperboliques.
\\Il en existe deux formes : 

\\- la forme conservative :

\begin{eqnarray}
\left\{ 
    \begin{array}{llll}
        u_t(t,x)+div_x(c(t,x)u(t,x))=0
        \\u(t=0,x)=u_0(x)
       	\end{array}
    \right .
\end{eqnarray}

\\- la forme forte : 

\begin{eqnarray}
\left\{ 
    \begin{array}{llll}
        u_t(t,x)+c(t,x)\nabla_xu(t,x)=d(t,x,u)
        \\u(t=0,x)=u_0(x)
       	\end{array}
    \right .
\end{eqnarray}

\\Dans notre cas, nous utilisons la forme conservative de l'équation de transport car la divergence du vecteur vitesse $c$ est nulle. L'intégrale de $u$  est conservée au cours du temps. 
\\
\\Pour résoudre le problème, il est nécessaire d'imposer une condition initiale à $t=0$ : \\$u(t=0,x)=u_0(x)$.
\\De plus, si $\Omega$ possède des bords il faut pour que le problème soit bien posé, ajouter une condition $u_{b}$ sur la partie de la frontière où le champ est rentrant ie :
\\
\begin{eqnarray}
\forall t\in \mathbb{R}_{+}, \forall x\in \partial\Omega,\;\;c(t,x).\overrightarrow{n(x)}\,<\,0\; \Rightarrow\, u(t,x) = u_b(t,x)
\end{eqnarray}
\\
Ou $\overrightarrow{n(x)}$ représente la normale extérieure à $\Omega$.
\\Ceci permet d'éviter au problème d'être sur-déterminé et de vérifier ainsi les trois conditions nécessaires qui définissent un problème bien posé (existence,unicité et stabilité).
\\ 
\\2. Dans le cas 1D avec un champ de vitesse constant $c$ le problème s'écrit sous la forme :
\begin{eqnarray}
    \left\{ 
    \begin{array}{llll}
        \partial_tu(t,x) + c\partial_xu(t,x)=0
        \\u(0,x)=u_0(x)
        \end{array}
    \right .
\end{eqnarray}
\\
Une des méthodes pour résoudre ce type d'équation est la méthode des caractéristiques. On définit les caractéristiques comme les courbes de $\mathbb{R}^2$ définies par $(t,X(t))$ ou $X(t)$ est la solution de l'équation différentielle ordinaire $\partial_tX(t)=c$.
On a alors que $(t,X(t))$ vérifie :
\\
\begin{eqnarray}
   \left\{ 
    \begin{array}{llll}
        \partial_tu(t,X(t))=\partial_tX*\partial_xu+\partial_tu
        =c*\partial_xu+\partial_tu =0
        \end{array}
    \right .
\end{eqnarray}
\\
Les solutions sont donc constantes le long des caractéristiques. 
Considérons $(t^*,x^*)$ un point du plan avec $t^*>0$ ainsi que $X^*(t)$ la caractéristique passant par ce point. Alors $X^*$ verifie :
\begin{eqnarray}
        \partial_tX^*=c \;,\;  X^*(t^*)=x^*
        \\\Rightarrow\ X^*=ct+x^*-ct^*\; et\;donc\; X^*(0)=x*-ct^*
\end{eqnarray}
Finalement au point $(t^*,x^*)$ sachant que la variable t ne rentre pas en compte car $u$ est contante le long des caractéristique, la solution $u$ du problème est déterminé par $u(t^*,x^*)=u(0,X^*(0))=u(0,x^*-ct^*)=u_0(x^*-ct^*)$
\\
\\On obtient finalement le critére suivant :
\\
\\ Si $u_0$ dérivale sur $\mathbb{R}$, alors il existe pour le probème de transport en une dimension une unique solution différentiable $u$ en $(t,x)$  donné par :

\begin{eqnarray}
        u(t,x)=u_0(x-ct) \;\;\;\;\forall x\in \mathbb{R}, \forall t>0 
\end{eqnarray}

3.
\section{Bibliographie}

Lien du sujet : https://github.com/upici/SaintVenant
Support de Recherche : http://www.i2m.univ-amu.fr/perso/maxime.hauray/enseignement/M2-Transport/Cours2.pdf
\\
\\Livres BU :
\\- Résolution numérique des équations de Saint-Venant par la technique de projection en utilisant une méthode des volumes finis dans un maillage non structuré 
\\
\\Se renseigner sur :
\\- méthode des volumes finis
\end{document}

